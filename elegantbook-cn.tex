\documentclass[lang=cn,10pt]{elegantbook}
\usepackage{yhmath}
\usepackage{float}
\usepackage{amsmath} 
\DeclareMathOperator{\arccot}{arccot}
\title{ElegantBook:优美的 \LaTeX{} 书籍模板}
\subtitle{Elegant\LaTeX{} 经典之作}

\author{Ethan Deng \& Liam Huang}
\institute{Elegant\LaTeX{} Program}
\date{April 9, 2022}
\version{4.3}
\bioinfo{自定义}{信息}

\extrainfo{不要以为抹消过去,重新来过,即可发生什么改变。—— 比企谷八幡}

\setcounter{tocdepth}{3}

\logo{logo-blue.png}
\cover{cover.jpg}

% 本文档命令
\usepackage{array}
\newcommand{\ccr}[1]{\makecell{{\color{#1}\rule{1cm}{1cm}}}}

% 修改标题页的橙色带
% \definecolor{customcolor}{RGB}{32,178,170}
% \colorlet{coverlinecolor}{customcolor}

\begin{document}

\maketitle
\frontmatter


\mainmatter


\chapter*{2025年秋季学期海德学院《常微分方程》第一次测试习题解答}

\begin{introduction}
  \item 第一章内容:微分方程概论
  \item 第二章内容:一阶微分方程的初等解法
\end{introduction}


\section*{一、是非题(本题10分,每小题2分,对的打√,错的打×)}

\begin{proposition*}{题目 1}
微分方程通解包含方程的所有解(\quad)
\end{proposition*}

\begin{note}
错
\end{note}

\begin{solution}
通常所称“通解”是含足够个任意常数的解族,并不必包含\emph{奇解}。如 Clairaut 方程 $y=Cx+\!C^{2}$ 的包络 $y=-x^{2}/4$ 是奇解,不在通解族内。因此“通解包含所有解”并不总成立。
\end{solution}


\begin{proposition*}{题目 2}
方程 $(y'')^{3}+\cos y'=2x$ 是 $3$ 阶微分方程(\quad)
\end{proposition*}

\begin{note}
错
\end{note}

\begin{solution}
微分方程的\emph{阶}由出现的最高阶导数决定。该式出现的最高阶导数是 $y''$,故为\textbf{二阶}方程;立方不改变阶数。
\end{solution}

\begin{proposition*}{题目 3}
线性微分方程 $\dfrac{dy}{dx}=P(x)y+Q(x)$ 的所有解构成线性空间(\quad)
\end{proposition*}

\begin{note}
错
\end{note}

\begin{solution}
非齐次线性方程的解集是\emph{仿射空间}:若 $y_1,y_2$ 是解,则一般 $\alpha y_1+\beta y_2$ 不是解(除非 $\alpha+\beta=1$)。只有齐次方程 $y'=P(x)y$ 的解集才构成线性空间。
\end{solution}

\begin{proposition*}{题目 4}
微分方程的积分因子是唯一的(\quad)
\end{proposition*}

\begin{note}
错
\end{note}

\begin{solution}
若 $\mu(x)$ 是一阶线性方程的一个积分因子,则对任意常数 $C\neq 0$,$C\mu(x)$ 也是积分因子。因此积分因子仅在“乘一个非零常数”的意义下唯一。
\end{solution}

\begin{proposition*}{题目 5}
$y=C_{1}e^{\,C_{2}-3x}-1$ 是 $y''-9y=9$ 的通解,其中 $C_{1},C_{2}$ 是任意常数(\quad)
\end{proposition*}

\begin{note}
错
\end{note}

\begin{solution}
代入可得 $y''-9y\equiv 9$,所给表达式确为\emph{一族}解。但 $C_{1}e^{C_{2}}$ 可合并为单一常数 $C$,于是只含一个有效常数:
\[
y=Ce^{-3x}-1.
\]
而二阶常系数非齐次方程的通解应为
\[
y=Ae^{3x}+Be^{-3x}-1,\qquad A,B\in\mathbb{R}.
\]
题设形式只覆盖其中 $A=0$ 的子族,故不是通解。
\end{solution}


\section*{二、选择题(本题15分,每小题3分)}

\begin{proposition*}{题目 1}
方程(\quad)是常微分方程。
\begin{flalign*}
(A)\;& y^{2}+y=3;&\\
(B)\;& y^{2}+y''=y';&\\
(C)\;& xy'+y=(xy)';&\\
(D)\;& x+z'_x+z'_y=y.&
\end{flalign*}
\end{proposition*}

\begin{note}
B
\end{note}

\begin{solution}
常微分方程是指未知函数为单变量 $y=y(x)$,且只含常导数的方程。\\[3pt]
(A) 不含导数,不是微分方程;\\
(C) 对等式右边化简后发现,$x'y = y$,不是常微分方程;\\
(D) 含偏导数 $z'_x, z'_y$,属于偏微分方程。
\end{solution}



\begin{proposition*}{题目 2}
已知方程 $\displaystyle \frac{dy}{dx}=\frac{y}{2x-y^{2}}$,它是(\quad)
\begin{flalign*}
(A)\;& \text{一阶线性微分方程},&\\
(B)\;& \text{齐次方程},&\\
(C)\;& \text{全微分方程},&\\
(D)\;& \text{变量分离方程}.&
\end{flalign*}
\end{proposition*}

\begin{note}
A
\end{note}

\begin{solution}
将 $x$ 视作 $y$ 的函数,有
\[
\frac{dx}{dy}=\frac{2x-y^{2}}{y}=\frac{2}{y}x - y,
\]
这是关于 $x(y)$ 的一阶线性方程,因此选 A。  
它既不是齐次方程,也不是全微分方程或可分离变量方程。
\end{solution}
\begin{theorem*}{视野拓展角——转换视角看问题}
当题目中没有明确说明谁是因变量,谁是自变量时,不要受到过往的学习经验主义影响,换个视角看问题可能会有不同的结果
\end{theorem*}
\begin{proposition*}{题目 3}
曲线 $x^{2}+y^{2}=C$ 满足的常微分方程是(\quad)
\begin{flalign*}
(A)\;& xy'+y=0,&\\
(B)\;& y'+\dfrac{x}{y}=0,&\\
(C)\;& y'+xy=0,&\\
(D)\;& yy'+2x=0.&
\end{flalign*}
\end{proposition*}

\begin{note}
B
\end{note}

\begin{solution}
对 $x^{2}+y^{2}=C$ 两边对 $x$ 求导:
\[
2x+2yy'=0 \Rightarrow yy'+x=0 \Rightarrow y'+\frac{x}{y}=0.
\]
因此选 B
\end{solution}

\begin{proposition*}{题目 4}
已知 $y(x)$ 满足 $xy'=y\ln\!\frac{y}{x}$,且当 $x=1$ 时 $y=e^{2}$。求当 $x=-1$ 时 $y=$(\quad)
\begin{flalign*}
(A)\;& -1,&\\
(B)\;& 0,&\\
(C)\;& 1,&\\
(D)\;& e^{-1}.&
\end{flalign*}
\end{proposition*}

\begin{note}
A
\end{note}

\begin{solution}
令 $v=\dfrac{y}{x}$,则 $y=vx,\ y'=v+xv'$。代入得:
\[
x(v+xv')=vx\ln v \Rightarrow v+xv'=v\ln v \Rightarrow xv'=v(\ln v-1).
\]

分离变量:$\displaystyle \frac{dv}{v(\ln v-1)}=\frac{dx}{x}$。  
解得:$\ln v=cx+1$。  

由初值 $x=1,v=e^{2}$ 得 $C=1$。于是$y=xe^{x+1}$。 

当 $x=-1$ 时,$\ln v=0 \Rightarrow v=1$,故 $y=vx=-1$。
\end{solution}
\begin{remark}
  题设微分方程仅在 $x\neq 0$ 且$x,y$同号时有意义,现在要求解在 $x=-1$ 处的值,因此隐含假设 $y(-1)<0$。根据这个也可求出解。
\end{remark}
\begin{proposition*}{题目 5}
积分方程 $\displaystyle y(x)=1+\int_{0}^{x}t\,y(t)\,dt$ 的解是(\quad)
\begin{flalign*}
(A)\;& y=e^{\frac{1}{2}x^{2}},&\\
(B)\;& y=e^{x^{2}},&\\
(C)\;& y=0,&\\
(D)\;& y=1.&
\end{flalign*}
\end{proposition*}

\begin{note}
A
\end{note}

\begin{solution}
对方程两边求导得 $y'(x)=x\,y(x)$,且 $y(0)=1$。  
分离变量:
\[
\frac{y'}{y}=x \Rightarrow \ln y=\frac{x^{2}}{2}+C.
\]
由初值 $y(0)=1$ 得 $C=0$,因此
\[
y(x)=e^{\frac{1}{2}x^{2}}.
\]
\end{solution}


\section*{三、填空题(本题25分,每题5分)}

\begin{proposition*}{题目 1}
以 \(xy=C_{1}e^{x}+C_{2}e^{-x}\) 为通解的常微分方程是\underline{\hspace{8em}}.
\end{proposition*}

\begin{note}
 \(x\,y''+2y'-x\,y=0\).
\end{note}

\begin{solution}
令 \(u(x)=x\,y(x)\),则题设给出 \(u=C_{1}e^{x}+C_{2}e^{-x}\),
故 \(u''-u=0\)。又
\[
u'=xy'+y,\qquad u''=xy''+2y'.
\]
代回得 \(xy''+2y'-xy=0\)。
\end{solution}

\begin{proposition*}{题目 2}
含两个任意常数 \(C_1,C_2\) 的函数 \(y=\varphi(x,C_1,C_2)\) 是二阶微分方程 
\(F(x,y,y',y'')=0\) 的通解的充要条件为\underline{\hspace{8em}}.
\end{proposition*}

\begin{note}
\(\displaystyle \frac{\partial(y,\,y')}{\partial(C_1,\,C_2)}\neq0\)且$y$代入后方程$F$恒成立\ \textbf{或}\ rank(\(\displaystyle \frac{\partial(y,\,y')}{\partial(C_1,\,C_2)}\))=2\ \textbf{或}\ $y$满足方程$F$且$C_1,C_2$相互独立且可唯一确定

(得分点在"满足方程"与“独立且唯一”的叙述,因为本题考查的就是让你叙述存在唯一性定理,因此仅说“满足存在唯一性定理”不得分)
\end{note}

\begin{solution}
通解能唯一确定 \(C_1,C_2\) 与 \((y,y')\) 的一一对应。
其充要条件是雅可比行列式
\[
\frac{\partial(y,\,y')}{\partial(C_1,\,C_2)}
=
\begin{vmatrix}
\partial y/\partial C_1 & \partial y/\partial C_2\\[2pt]
\partial y'/\partial C_1 & \partial y'/\partial C_2
\end{vmatrix}\neq0,
\]
这样可由 \(y,y'\) 解出 \(C_1,C_2\),从而消去常数得到二阶方程。
\end{solution}


\begin{proposition*}{题目 3}
与曲线族 \(y=C\,e^{x}\) 正交的曲线族为\underline{\hspace{8em}}.
\end{proposition*}

\begin{note}
\(2x+y^{2}=C\)
\end{note}

\begin{solution}
对 \(y=C e^{x}\) 有切线斜率 \(y'=y\)。正交轨迹的斜率满足
\(y'\cdot y'_{\perp}=-1\),故 \(y'_{\perp}=-1/y\)。
积分得
\[
y\,dy=-dx \;\Rightarrow\; \tfrac12 y^{2}+x=C \;\Rightarrow\; 2x+y^{2}=C.
\]
\end{solution}


\begin{proposition*}{题目 4}
一阶线性微分方程 \(\displaystyle \frac{dy}{dx}=P(x)y+Q(x)\) 的积分因子是\underline{\hspace{8em}}.
\end{proposition*}

\begin{note}
 \(\displaystyle \mu(x)=e^{-\int P(x)\,dx}\)
\end{note}

\begin{solution}
标准一阶线性方程写作
\[
\frac{dy}{dx}-P(x)y=Q(x),
\]

此时积分因子应满足
\[
\frac{d\mu}{dx}=-P(x)\mu.
\]

积分得
\[
\mu(x)=e^{-\int P(x)\,dx}.
\]

两边同乘以 \(\mu(x)\),得
\[
\mu\,y'-\mu P\,y=(\mu y)'=\mu Q(x),
\]

从而左端成为全导数形式。这样便可积分得到通解。
\end{solution}


\begin{proposition*}{题目 5}
一阶微分方程 \((2x+y-4)\,dx+(x+y-1)\,dy=0\) 的通解为\underline{\hspace{8em}}.
\end{proposition*}

\begin{note}
\(x^{2}+xy-4x+\tfrac12 y^{2}-y=C\)或$(x+y-1)^2+(x-3)^2=C$
\end{note}

\begin{solution}


\textbf{1) 平移消常数:} 取
\[
x=X+3,\qquad y=Y-2 \quad(\text{即 }X=x-3,\;Y=y+2),
\]

则
\[
(2x+y-4)\,dx+(x+y-1)\,dy=(2X+Y)\,dX+(X+Y)\,dY=0.
\]

这是一阶\emph{齐次}方程。

\textbf{2) 令 } \(Y=vX\),则 \(dY=v\,dX+X\,dv\)。代入得
\[
(2X+vX)\,dX+(X+vX)\,(v\,dX+X\,dv)=0
\]

化简为
\[
\bigl(v^{2}+2v+2\bigr)\,dX+(1+v)X\,dv=0
\quad\Rightarrow\quad
\frac{dX}{X}=-\frac{1+v}{v^{2}+2v+2}\,dv.
\]

\textbf{3) 分离并积分:} 记 \(u=v+1\),则 \(v^{2}+2v+2=u^{2}+1\),
\[
\ln|X|=-\int\frac{u}{u^{2}+1}\,du
=-\tfrac12\ln(u^{2}+1)+C,
\]

从而
\[
|X|\sqrt{(v+1)^{2}+1}=C_1.
\]

\textbf{4) 还原 } \(v=\dfrac{Y}{X}\):
\[
|X|\,\sqrt{\Bigl(\frac{Y}{X}+1\Bigr)^{2}+1}
=\sqrt{X^{2}+(X+Y)^{2}}=C_1
\ \Rightarrow\ 
X^{2}+(X+Y)^{2}=C_2.
\]

\textbf{5) 换回原变量:} \(X=x-3,\;Y=y+2\),得
\[
(x-3)^{2}+(x+y-1)^{2}=C_2
\ \Longleftrightarrow\ 
x^{2}+xy-4x+\tfrac12 y^{2}-y=C,
\]


\end{solution}
\begin{theorem*}{视野拓展角——利用分项组合求解恰当微分方程}
上述求解可以看出直接用变量替换和分离变量求解是非常繁琐的。对于恰当微分方程的求解,在熟练记忆了一些常见的一些简单的二元函数的全微分的基础上,可以用分项组合的方法求解,常见的二元函数全微分有:
\begin{align*}
& y\,dx + x\,dy = d(xy), \\[6pt]
& y\,dx - x\,dy = -\,d\!\left(\frac{y}{x}\right) \,x^{2}, \\[6pt]
& \frac{y\,dx - x\,dy}{x^{2} + y^{2}} = d\!\left(\arccot\frac{y}{x}\right), \\[6pt]
& \frac{y\,dx - x\,dy}{y^{2}} = d\!\left(\frac{x}{y}\right), \\[6pt]
& \frac{y\,dx - x\,dy}{x\,y} = d\!\left(\ln\frac{x}{y}\right), \\[6pt]
& \frac{y\,dx - x\,dy}{x^{2} - y^{2}} = \tfrac{1}{2}\,d\!\left(\ln\left|\frac{x - y}{x + y}\right|\right).
\end{align*}

\noindent\rule{\textwidth}{0.4pt}
接下来,我们用分项组合方法求解本题:
分项并配成常见全微分:
\[
\underbrace{2x\,dx}_{d(x^{2})}
+\underbrace{(y\,dx+x\,dy)}_{d(xy)}
+\underbrace{y\,dy}_{d\!\big(\tfrac12 y^{2}\big)}
+\underbrace{(-4\,dx-1\,dy)}_{d(-4x-y)}=0.
\]

于是
\[
d\!\left(x^{2}+xy+\tfrac12 y^{2}-4x-y\right)=0,
\]

积分得隐式通解
\[
x^{2}+xy+\tfrac12 y^{2}-4x-y=C
\]
\end{theorem*}


\section*{四、求微分方程通解(本题32分,每小题8分)}

\begin{proposition*}{题目 1}

\[
y'=\frac{x}{\cos y}-\tan y.
\]

\end{proposition*}

\begin{note}
\(\displaystyle \sin y=x-1+C\,e^{-x}\)(或 \(y=\arcsin(x-1+C\,e^{-x})\))
\end{note}

\begin{solution}
两边同乘以 \(\cos y\):
\[
(\sin y)'=x-\sin y.
\]

设 \(z=\sin y\),得线性方程 \(z'+z=x\)。积分因子 \(e^{x}\),
\[
(z e^{x})' = x e^{x}\ \Rightarrow\ z e^{x} = \int x e^{x}\,dx=(x-1)e^{x}+C.
\]

故 \(z=\sin y=x-1+C e^{-x}\)。
\end{solution}



\begin{proposition*}{题目 2}
\[
\int_{0}^{x}y(t)\,dt+\int_{0}^{x}(x-t)\,[\,2t\,y(t)+t\,y^{2}(t)\,]\,dt=x.
\]

\end{proposition*}

\begin{note}
\(\displaystyle y(x)=\frac{2}{3e^{x^{2}}-1}\)或$\frac{y}{y+2}=\frac{1}{3}e^{-x^2}$或$\ln|\frac{y}{y+2}|=-x^2-\ln3$
\end{note}

\begin{solution}
设左端为 \(F(x)\)。微分得
\[
F'(x)=y(x)+\int_{0}^{x}[\,2t\,y(t)+t\,y^{2}(t)\,]\,dt=1.
\]

再微分一次:
\[
y'(x)+\bigl(2xy(x)+x\,y^{2}(x)\bigr)=0,
\]

即 \(y'=-x\,y(2+y)\)。变量分离:
\[
\frac{dy}{y(2+y)}=-x\,dx
\Rightarrow \ln\!\frac{|y|}{|2+y|}=-x^{2}+C.
\]

化为 \(\displaystyle y=\frac{2}{A e^{x^{2}}-1}\)。由
\(F'(0)=1\Rightarrow y(0)=1\),得 \(A=3\),故
\(\displaystyle y(x)=\frac{2}{3e^{x^{2}}-1}\)。
\end{solution}


\begin{proposition*}{题目 3}
\[
\frac{dy}{dx}=\frac{4x^{3}-2xy^{3}+2x}{3x^{2}y^{2}-6y^{5}+3y^{2}}.
\]

\end{proposition*}

\begin{note}
 \(\displaystyle (3y^{3}-1)^{2}-(3y^{3}-1)(3x^{2}+1)+(3x^{2}+1)^{2}=C\)
\end{note}

\begin{solution}




\textbf{(1) 令新变量 \(u=y^{3}\), \(v=x^{2}\)。}

由于
\[
\frac{du}{dx}=3y^{2}\frac{dy}{dx},\qquad \frac{dv}{dx}=2x,
\]

原方程可化为
\[
\frac{du}{dv}=\frac{3y^{2}\,dy/dx}{2x}
=\frac{3y^{2}\bigl(4x^{3}-2xy^{3}+2x\bigr)}{2x\bigl(3x^{2}y^{2}-6y^{5}+3y^{2}\bigr)}
=\frac{2v-u+1}{v-2u+1}.
\]

于是得到关于 \(u,v\) 的一阶方程:
\[
\frac{du}{dv}=\frac{2v-u+1}{v-2u+1}.
\]



\textbf{(2) 变换变量使方程齐次化。}

由方程组
\[
\begin{cases}
2v-u+1=0,\\
v-2u+1=0
\end{cases}
\Rightarrow u=\tfrac{1}{3},\;v=-\tfrac{1}{3}.
\]

取平移变换:
\[
U=u-\tfrac{1}{3},\qquad V=v+\tfrac{1}{3},
\]

代入原式得
\[
\frac{dU}{dV}=\frac{2V-U}{V-2U}.
\]



\textbf{(3) 令 \(S=U/V\),化为齐次方程。}

有 \(U=SV,\; dU=S\,dV+V\,dS\),代入得:
\[
S+\frac{V\,dS}{dV}=\frac{2V-SV}{V-2SV}
\Rightarrow \frac{dS}{dV}=-\frac{1}{2V}\cdot\frac{2S^{2}-2S+2}{1-2S}.
\]



\textbf{(4) 分离变量积分。}

移项得
\[
-\frac{1}{2}\frac{2S-1}{S^{2}-S+1}\,dS=\frac{dV}{V}.
\]

两边积分,得
\[
\ln V = -\frac{1}{2}\ln(S^{2}-S+1)+C
\quad\Rightarrow\quad (S^{2}-S+1)V^{2}=C.
\]



\textbf{(5) 还原变量。}
由 \(S=\dfrac{U}{V}\),得
\[
U^{2}-U V+V^{2}=C.
\]

再代回 \(U=u-\tfrac{1}{3},\;V=v+\tfrac{1}{3}\) 及 \(u=y^{3},v=x^{2}\),得到
\[
(3y^{3}-1)^{2}-(3y^{3}-1)(3x^{2}+1)+(3x^{2}+1)^{2}=C.
\]

这就是原方程的隐式通解。
\end{solution}

\begin{theorem*}{视野拓展角——势函数法求解恰当微分方程}
上述可以看出,此题用变量替换后分离变量会很麻烦,他又恰好是恰当微分方程,但是又不容易“一眼”看出来每一部分的原函数,因此我们可以用势函数求解法简化

设微分方程
\[
M(x,y)\,dx+N(x,y)\,dy=0.
\]
\begin{enumerate}[label=\textbf{步骤 \arabic*:}, leftmargin=4em]
  \item \textbf{检验恰当性}\; 计算
  \[
  M_y:=\frac{\partial M}{\partial y},\qquad N_x:=\frac{\partial N}{\partial x}.
  \]
  若 \(M_y=N_x\),则方程恰当,存在势函数 \(F(x,y)\) 使 \(dF=M\,dx+N\,dy\)。

  \item \textbf{先对 \(x\) 积分构造势函数}
  \[
  F(x,y)=\int M(x,y)\,dx\;+\;h(y),
  \]
  其中 \(h(y)\) 为“积分常数函数”。

  \item \textbf{用 \(F_y=N\) 求 \(h'(y)\)}
  \[
  F_y=\frac{\partial}{\partial y}\!\Big(\int M\,dx\Big)+h'(y)=N(x,y)
  \ \Longrightarrow\ 
  h'(y)=N(x,y)-\frac{\partial}{\partial y}\!\Big(\int M\,dx\Big).
  \]

  \item \textbf{积分得到 \(h(y)\) 并代回}
  \[
  h(y)=\int h'(y)\,dy,\qquad
  F(x,y)=\int M\,dx+h(y).
  \]

  \item \textbf{给出隐式解}
  \[
 F(x,y)=C,\qquad(\text{必要时可解出显式 }y(x)).
  \]
\end{enumerate}
\noindent\rule{\textwidth}{0.4pt}
下面我们用上述的势函数求解方法求解本题:

把方程改写为微分形式
\[
(3x^{2}y^{2}-6y^{5}+3y^{2})\,dy-(4x^{3}-2xy^{3}+2x)\,dx=0,
\]
记
\[
M(x,y)=-(4x^{3}-2xy^{3}+2x),\qquad
N(x,y)=3x^{2}y^{2}-6y^{5}+3y^{2}.
\]

\textbf{(1)恰当性:}
\[
M_y=6xy^{2},\qquad N_x=6xy^{2}\quad\Rightarrow\quad M_y=N_x,
\]

因此为恰当微分方程,存在势函数 \(F\) 使 \(dF=M\,dx+N\,dy\)。

\textbf{(2)先对 \(x\) 积分求 \(F\):}
\[
F(x,y)=\int M\,dx
= -x^{4}+x^{2}y^{3}-x^{2}+h(y).
\]

\textbf{(3) 由 \(F_y=N\) 求 \(h(y)\):}
\[
F_y=3x^{2}y^{2}+h'(y)=3x^{2}y^{2}-6y^{5}+3y^{2}
\ \Rightarrow\ h'(y)=-6y^{5}+3y^{2}.
\]
积分得 \(h(y)=-y^{6}+y^{3}\)(常数并入总常数)。

\textbf{(4) 写出势函数与通解:}
\[
F(x,y)=-x^{4}+x^{2}y^{3}-x^{2}-y^{6}+y^{3}=C,
\]
等价整理为
\[
x^{4}-x^{2}y^{3}+x^{2}+y^{6}-y^{3}=C
\]

\end{theorem*}

\begin{proposition*}{题目4}
\[
y=3xy'+y^{2}(y')^{2}
\]
\end{proposition*}

\begin{note}
通解为$y^{3}=3Cx+C^{2}$或者($x= \frac{\sqrt{C/p}-Cp}{3p},y = \sqrt{\frac{C}{p}}$)或者($x= \frac{y-y^2p^2}{3p},y = \sqrt{\frac{C}{p}}$),特解为$9x^{2}+4y^{3}=0$
\end{note}

\begin{solution}
记 \(p=y'\)。原式写为
\[
y-3xp-y^{2}p^{2}=0. \tag{1}
\]

两边对 \(x\) 求导(用 \(p'=dp/dx\)):
\[
p-3(p+xp')-(2yp^{3}+2y^{2}pp')=0
\;\Rightarrow\;
(-3x-2y^{2}p)p'=2p\,(1+yp^{2}). \tag{2}
\]

利用(1) 把 \(x\) 用 \(y,p\) 表示:\(x=\dfrac{y-y^{2}p^{2}}{3p}\)。代入(2) 得
\[
\frac{dp}{dx}\Bigl(-\frac{y}{p}-y^{2}p\Bigr)=2(1+yp^{2}).
\]

又 \(dp/dx = (dp/dy)\,y'=(dp/dy)\,p\),于是
\[
\frac{dp}{dy}\cdot y\Bigl(\frac{1+yp^{2}}{p}\Bigr)=-2(1+yp^{2})
\ \Rightarrow\
\frac{1}{p}\,dp=-2\,\frac{dy}{y}.
\]

分离变量积分得
\[
\ln|p|=-2\ln|y|+C_0\quad\Rightarrow\quad p=\frac{C}{y^{2}}\quad(C=e^{C_0}).
\]

回代 \(p=y'=\dfrac{C}{y^{2}}\),分离变量:
\[
y^{2}\,dy=C\,dx
\ \Rightarrow\
\frac{1}{3}y^{3}=Cx+C_1
\ \Rightarrow\
y^{3}=3Cx+C^{2},
\]

其中把常数重命名为同一参数 \(C\)(因式配方可吸收 \(C_1\))。
因此得到通解
\[
y^{3}=3Cx+C^{2}
\]

当 \(C=0\) 时给出解 \(y\equiv0\);此外还可由判别式为零得到包络(特解)\(9x^{2}+4y^{3}=0\)。
\end{solution}

\section*{五、求解下列问题(本题18分,每小题9分)}



\begin{proposition*}{题目 1}
有连接点 $O(0,0)$ 和 $A(1,1)$ 的一段向上凸的曲线弧 $\wideparen{OA}$,对于 $\wideparen{OA}$ 上任意点 $P(x,y)$,曲线弧 $\wideparen{OP}$ 与直线段 $\overline{OP}$ 所围图形的面积为 $x^{2}$,求曲线弧 $\wideparen{OA}$ 的方程。
\end{proposition*}

\begin{note}
$f(x)=x-4x\ln x$
\end{note}

\begin{solution}



设曲线方程为 $y=f(x)$。

由题意可知,曲线弧与弦所围成的面积为 $x^2$,即
\[
\int_{0}^{x} f(u)\,du - \tfrac{1}{2}x f(x) = x^{2}.
\]

对 $x$ 两边求导:
\[
f(x) - \tfrac{1}{2} f(x) - \tfrac{1}{2}x f'(x) = 2x,
\]

即
\[
f'(x) - \frac{f(x)}{x} = -4.
\]

这是一个一阶线性非齐次微分方程。对应齐次方程为
\[
f'(x) - \frac{f(x)}{x} = 0.
\]

分离变量:
\[
\frac{1}{f}df = \frac{1}{x}dx,
\]

两边积分:
\[
\ln|f| = \ln|x| + C \quad\Rightarrow\quad f = Cx.
\]

设齐次方程的通解为 $f_h=Cx$,则令 $f=C(x)x$,有
\[
f'(x)=C'(x)x+C(x).
\]

代入原方程:
\[
C'(x)x + C(x) - \frac{C(x)x}{x} = -4,
\]

化简得
\[
x C'(x) = -4.
\]

两边积分:
\[
C(x) = -4\ln x + c.
\]

于是通解为
\[
f(x) = (-4\ln x + c)x = -4x\ln x + cx.
\]

代入条件 $f(1)=1$,得
\[
1 = -4\cdot1\cdot\ln 1 + c \quad\Rightarrow\quad c=1.
\]

故曲线方程为
\[
f(x)=x-4x\ln x
\]

\begin{figure}[H]
  \centering
  \includegraphics[width=0.5\textwidth]{111.jpg}
  \caption{题目示意图}
\end{figure}
\end{solution}


\begin{proposition*}{题目2}
求具有性质
\[
f(x+t)=\frac{f(x)+f(t)}{1-f(x)f(t)}\qquad(\forall\,x,t\in\mathbb R),
\]
的函数$f(x)$且, $f'(0)$ 存在.
\end{proposition*}

\begin{note}
$f(x)=\tan\!\big(c\,x\big)$,其中常数 $c=f'(0)$。
\end{note}

\begin{solution}
由题设关系式 
\[
f(t+s)=\frac{f(t)+f(x)}{1-f(t)f(x)},
\]

令 \(x=0\),得 
\[
x(t)=\frac{f(t)+f(0)}{1-f(t)f(0)}.
\]

由于 \(f(0)=0\),上式恒成立。

接着考虑导数定义:
\[
f'(t)=\lim_{\Delta t\to0}\frac{f(t+\Delta t)-f(t)}{\Delta t}.
\]

根据函数方程:
\[
f(t+\Delta t)=\frac{f(t)+f(\Delta t)}{1-f(t)f(\Delta t)}.
\]

因此
\[
f(t+\Delta t)-f(t)=\frac{[1+f^2(t)]f(\Delta t)}{1-f(t)f(\Delta t)}.
\]

因为 \(f'(0)\) 存在且 \(f(0)=0\),有
\[
\lim_{\Delta t\to0}\frac{f(\Delta t)}{\Delta t}=f'(0),
\]

从而
\[
f'(t)=\lim_{\Delta t\to0}\frac{f(t+\Delta t)-f(t)}{\Delta t}
      =\lim_{\Delta t\to0}\frac{[1+f^2(t)]x(\Delta t)}{(1-f(t)f(\Delta t))\Delta t}
      =[1+f^2(t)]f'(0).
\]

由初值条件 \(f'(0)=1\),可得
\[
f'(t)=1+f^2(t)
\]

分离变量、积分:
\[
\int\frac{dx}{1+x^2}=\int dt
\quad\Rightarrow\quad
\arctan f(t)=t+c.
\]

由 \(f(0)=0\Rightarrow c=0\),
故
\[
f(t)=\tan t
\]

更一般地,若 \(f'(0)=c\),则解为
\[
f(t)=\tan(c\,t)
\]
\end{solution}

\begin{theorem*}{视野拓展角——异题同构}
此题非常经典,课本课后题也有所涉及,利用导数的定义并结合常微分方程的求解这种抽象函数的方法需要同学们掌握,希望同学们认真体会以下几个题的求解方法,并能举一反三。
\begin{enumerate}

  \item 已知函数满足
  $
  f(x+t)=\frac{f(x)+f(t)}{\,1+f(x)f(t)\,}.
  $
  \begin{solution}
  \(f(x)=\tanh(cx)\)。
  \end{solution}


  \item 已知函数满足
  $
  f(x+t)=f(x)f(t).
  $
  \begin{solution}
  \(f(x)=e^{cx}\)。
  \end{solution}

  \item 已知函数满足
  $
  f(x+t)=f(x)+f(t).
  $
  \begin{solution}
  \(f(x)=cx\)。
  \end{solution}




\end{enumerate}
\end{theorem*}


\end{document}
